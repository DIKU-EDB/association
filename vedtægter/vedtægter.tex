% vim: spell spelllang=da_dk

\documentclass[a4paper]{article}

% Tekstindkodning og orddeling
\usepackage[utf8]{inputenc}
\usepackage[danish, english]{babel}

% Palatino skrifttype
\usepackage[T1]{fontenc}
\usepackage{mathpazo}

% Brug enumerate, med disse macro'er ved hånden
\usepackage{enumitem}
% afsnitenum er erklæret her for lokalitet, men kun brugt 1 gang da
% enumereringen skal fortsætte på tværs af afsnit.
\newenvironment{afsnitenum}{
  \begin{enumerate}[series=afsnit, %
    label=\textsection~\arabic*., ref=\textsection~\arabic*]%
}{\end{enumerate}}
\newenvironment{stykenum}{
  \begin{enumerate}[%
    label=Stk.~\arabic*., ref=\theenumi~Stk.~\arabic*, start=2]
}{\end{enumerate}}

% Fint hoved og fod
\usepackage{fancyhdr}
\usepackage{lastpage}
\renewcommand{\headrulewidth}{0in}
\renewcommand{\headsep}{40pt}
\setlength{\headheight}{20pt}
\pagestyle{fancy}
\cfoot{\thepage/\pageref{LastPage}}
\fancypagestyle{first}{%
  \fancyhf{}%
  \cfoot{\thepage/\pageref{LastPage}}%
}


\title{Vedtægter for DIKUNIX}
\author{Datalogisk institut, Københavns Universitet}
\date{Som vedtaget den 9. december 2016.}


\begin{document}

\maketitle
\thispagestyle{first}

\section*{Navn, formål og hjemsted}

\begin{afsnitenum}

\item Foreningens navn er "DIKUNIX".

  \begin{stykenum}

  \item Foreningens hjemsted er Datalogisk institut ved Københavns Universitet
        (DIKU), Københavns Kommune.

  \end{stykenum}

\item Foreningens formål er at opsætte og vedligeholde EDB-baserede tjenester
til støtte af sociale foreninger og studerende i almindelighed på DIKU.

  \begin{stykenum}

  \item Foreningen er apolitisk og uafhængig af partipolitiske,
        institutionel\-le og erhvervsmæssige interesser.

  \end{stykenum}

\end{afsnitenum}


\section*{Medlemskab}

\begin{enumerate}[resume*=afsnit]

\item Som medlem optages enhver, som er ansat eller studerende på DIKU.
Medlemskab tegnes for et studieår af gangen. Ved ophør af tilhørsforhold til
instituttet, bortfalder stemmeret øjeblikkeligt.  Medlemskab tegnes dog fortsat
for det efterfølgende studieår, og kan forlænges årligt efter videre aftale med
den siddende bestyrelse.

  \begin{stykenum}

  \item Der betales ikke kontingent, da foreningens drift skal hvile i sig
        selv.  Bestyrelsen har dog ret til at opkræve kontingent for fortsat
        medlemskab efter ophør af tilhørsforhold til instituttet, ud over
        det første efterfølgende studieår.

\end{stykenum}

\end{enumerate}


\section*{Generalforsamling}

\begin{enumerate}[resume*=afsnit]

\item Generalforsamlingen er foreningens øverste myndighed.

  \begin{stykenum}

  \item Der afholdes årligt en generalforsamling i efterårssemestret, indkaldt
        af bestyrelsen, hvor foreningens bestyrelse vælges.

  \item \label{stemmeberettigede} Stemmeberettigede er indskrevne studerende og
        ansatte ved DIKU.

  \item Hver af de i \ref{stemmeberettigede} nævnte bærer hver en stemme.

  \item Det er ikke tilladt at stemme med fuldmagt.

  \end{stykenum}

\item Der indkaldes til den årlige generalforsamling mindst fire uger før\\ afholdelse.

  \begin{stykenum}

  \item Indkaldelsen skal være offentlig og tilgængelig på DIKU.

  \item På generalforsamlingen kan der stilles ændringsforslag til\\vedtægter og
        forslag til arbejdsplan mm. og der kan opstilles\\kandidaturer til
        bestyrelsen.

  \item Forslag skal være bestyrelsen i hænde senest to uger før afholdelse af
        den årlige generalforsamling.

  \item Forslag og dagsorden skal gøres tilgængelig på foreningens webside
        senest to uger før afholdelse af generalforsamling.

  \item Ændringsforslag til forslag skal være bestyrelsen i hænde senest en uge
        før afholdelse af generalforsamlingen og gøres tilgængelig på
        foreningens webside.

  \item Ændringsforslag til ændringsforslag kan stilles mundtligt på
        generalforsamlingen.

  \end{stykenum}

\item Dagsordnen til den årlige generalforsamling skal som minimum indeholde:

  \begin{itemize}
  \item Formalia
    \begin{itemize}
    \item Valg af dirigent.
    \item Valg af stemmetællere.
    \item Valg af referent.
    \item Godkendelse af dagsorden.
    \end{itemize}
  \item Indkommende forslag.
  \item Valg til bestyrelsen.
  \item Valg af revisor
  \item Godkendelse af regnskab.
  \item Evt.
  \end{itemize}

\end{enumerate}

\section*{Valg til bestyrelse}

\begin{enumerate}[resume*=afsnit]

\item Opstilling af kandidatur til bestyrelsen skal være bestyrelsen i hænde
senest to uger før afholdelse af generalforsamlingen og gøres tilgængelig på
foreningens webside mindst en uge før generalforsamlingen.

  \begin{stykenum}

  \item Opstillingsberettigede til bestyrelsen er indskrevne studerende og
        ansatte ved DIKU.

  \item Er der ikke anmeldt et tilstrækkeligt antal kandidater til bestyrelsen,
        afholdes først tillidsvalg om de rettidige opstillede kandidater,\\
        hvorefter det er muligt at opstille under mødet.

  \item Er antallet af anmeldte kandidater lig antallet af pladser afholdes
        tillidsvalg.

  \item Er antallet af anmeldte kandidater større end antallet af pladser,
        afholdes forholdstalsvalg, hvor hver stemmeberettiget har maksimalt lige
        så mange stemmer som antallet af bestyrelsespladser. Hver
        stemmeberettiget kan kun give en stemme per opstillet kandidat.

  \end{stykenum}

\item Forslag vedtages ved 2/3 flertal af de fremmødte.

  \begin{stykenum}

  \item Ændringsforslag til forslag vedtages ved simpelt flertal.

  \item Først stemmes om ændringsforslag og derefter om forslag.

  \end{stykenum}

\end{enumerate}


\section*{Ekstraordinær generalforsamling}

\begin{enumerate}[resume*=afsnit]

\item Hvis et simpelt flertal af bestyrelsens medlemmer eller mindst 10 af de
stemmeberettigede ønsker det, kan der indkaldes til en ekstraordinær
generalforsamling.

  \begin{stykenum}

  \item De ekstraordinære generalforsamlinger indkaldes og afholdes jf. § 4-8.

  \item Dog er den minimale dagsorden til ekstraordinære generalforsamlinger
        som følger:

        \begin{itemize}
        \item Formalia
          \begin{itemize}
          \item Valg af dirigent.
          \item Valg af stemmetællere.
          \item Valg af referent.
          \item Godkendelse af dagsorden.
          \end{itemize}
        \item Evt.
        \end{itemize}

  \end{stykenum}

\end{enumerate}


\section*{Bestyrelsens sammensætning}

\begin{enumerate}[resume*=afsnit]

\item Bestyrelsen består af op til 7 personer og vælges på den årlige
generalforsamling.

  \begin{stykenum}

  \item Bestyrelsens medlemmer vælges for 1 år ad gangen.

  \end{stykenum}

\end{enumerate}


\section*{Bestyrelsens ansvar og arbejde}

\begin{enumerate}[resume*=afsnit]

\item Bestyrelsen har, mellem generalforsamlingerne, ansvaret for foreningen.

  \begin{stykenum}

  \item Bestyrelsen leder foreningen i overensstemmelse med nærværende
        vedtægter og generalforsamlingens beslutninger.

  \item Bestyrelsen træffer beslutninger med simpelt flertal.

  \item Bestyrelsen vælger imellem sig en formand og en kasserer.

  \item Bestyrelsen fastsætter selv sin forretningsorden. Den kan nedsætte
        underudvalg og arbejdsgrupper til varetagelse af afgrænsede opgaver.

  \end{stykenum}

\end{enumerate}


\section*{Økonomi, regnskab og revision}

\begin{enumerate}[resume*=afsnit]

\item Foreningens regnskabsår følger kalenderåret.

  \begin{stykenum}

  \item Bestyrelsen er ansvarlig overfor generalforsamlingen for budget\\samt
        regnskab.

  \item Foreningens regnskab føres af kassereren.

  \item Regnskabet revideres af de på generalforsamlingen valgte revisorer.

  \end{stykenum}

\end{enumerate}

\section*{Tegningsregler og hæftelse}

\begin{enumerate}[resume*=afsnit]

\item Foreningen tegnes udadtil ved underskrift af mindst to af bestyrelsens
medlemmer. Ved optagelse af lån og ved salg/ pantsætning af fast ejendom tegnes
foreningen af den samlede bestyrelse.

  \begin{stykenum}

  \item Der påhviler ikke foreningens medlemmer nogen personlig hæftelse for de
        forpligtelser, der påhviler foreningen.

  \end{stykenum}

\end{enumerate}


\section*{Vedtægtsændringer}

\begin{enumerate}[resume*=afsnit]

\item Disse vedtægter kan kun ændres med 2/3 flertal på en generalforsamling,
hvor ændringsforslaget fremgår af dagsordenen.

  \begin{stykenum}

  \item Ændringer i §§ 1, 2, 4, 14 og 15 kan kun vedtages med 4/5 af de
        tilstedeværende stemmer.

  \item Vedtægtsændringerne træder i kraft med virkning fra den
        generalforsamling, de vedtages på.

  \end{stykenum}

\end{enumerate}


\section*{Opløsning}

\begin{enumerate}[resume*=afsnit]

\item Opløsning af foreningen kan kun finde sted med 2/3 flertal på to hinanden
følgende generalforsamlinger, hvoraf den ene skal være den årlige. Tidsrummet
mellem de to generalforsamlinger skal være min. 1 måned.

  \begin{stykenum}

  \item Foreningens formue skal i tilfælde af opløsning anvendes i
        over\-ensstemmelse med de i § 2 fastsatte formål eller skænkes til andre
        studenterforeninger på DIKU. Beslutning om den konkrete anvendelse af
        formuen træffes af den opløsende generalforsamling.

  \end{stykenum}

\end{enumerate}


\bigskip

\begin{center}

\emph{Ovenstående vedtægter er vedtaget den 9. december 2016 på den stiftende
generalforsamling.}

\end{center}

\end{document}
